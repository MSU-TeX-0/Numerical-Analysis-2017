\documentclass[a4paper,10pt,fleqn]{article}
\usepackage[margin=1in]{geometry}
%\usepackage{xltxtra}
\usepackage{fontspec}
\usepackage{polyglossia}
\usepackage{mathtools}
\usepackage{commath}
\usepackage{unicode-math}
\usepackage{amsthm}
\usepackage{graphicx}
\usepackage{titlesec}
\usepackage[inline]{enumitem}

\title{Лекции по численным методам}
\author{}
%\date{}
\setmainfont{Nimbus Roman}
\setmathfont{texgyretermes-math.otf}
\setdefaultlanguage{russian}

%\titleformat{\section}{\normalfont\Large\bfseries}{Лекция \thesection}{1em}{}
\everymath{\displaystyle}

\renewcommand{\eval}{\Big\vert}
\newcommand{\const}{\operatorname{const}}
\DeclareMathOperator{\sgn}{sgn}
\DeclareMathOperator{\dist}{ρ}

\newcommand{\setw}[2]{\set{#1 \colon #2}}
\newcommand{\cvrg}[2]{\overset{#1}{\underset{#2}{\rightarrow}}}
\newcommand{\uncvrg}[2]{\overset{#1}{\underset{#2}{\rightrightarrows}}}
\newcommand{\uninscvrg}[2]{\uncvrg{\ins{#1}}{#2}}

% Inline list
\newlist{mylist}{enumerate*}{1}
\setlist[mylist]{label=(\arabic*)}

% Common environments
\theoremstyle{definition}
%\theoremstyle{plain}
\newtheorem{lemma}{Лемма}[section]
\newtheorem{theorem}{Теорема}[section]
\newtheorem{corollary}{Следствие}[section]
\theoremstyle{definition}
\newtheorem{definition}{Опредение}[section]
\newtheorem{proposition}{Предложение}[section]
\newtheorem{example}{Пример}[section]
\newtheorem{exercise}{Упражнение}[section]
\newtheorem*{exercise*}{Упражнение*}
\newtheorem{task}{Задача}[section]
\theoremstyle{remark}
\newtheorem*{note}{Замечание}

% Aliases
\newenvironment{df}{\begin{definition}}{\end{definition}}
\newenvironment{thm}{\begin{theorem}}{\end{theorem}}
\newenvironment{rk}{\begin{proposition}}{\end{proposition}}
\newenvironment{lm}{\begin{lemma}}{\end{lemma}}
\newenvironment{ex}{\begin{example}}{\end{example}}
\newenvironment{tk}{\begin{task}}{\end{task}}
\newenvironment{pf}{\begin{proof}}{\end{proof}}


\begin{document}
	\renewcommand{\setminus}{∖}
\abovedisplayskip=2mm
\belowdisplayskip=2mm
\setlength{\parindent}{0pt}
\setlength{\parskip}{3mm plus2mm minus1mm}
\newcommand{\eps}{\varepsilon}
\renewcommand{\phi}{\varphi}

	\maketitle

\section{Решение нелинейных уравнений}

\subsection{Деление отрезка пополам}
Самый тривиальный метод, какой только можно придумать. Но сходится, только очень медленно.

\subsection{Метод простой итерации}
Пусть дано уравнение $x = g(x)$, причём $g$ — сжимающее отображение (т. е. $\dist(g(x), g(y)) ≤ q\dist(x, y)$, где $q ∈ \intoo{0, 1}$). Рассмотрим последовательность $x^{n+1} = g(x^n)$ с начальным условием $x^0$. Она сходится к решению уравнения.
\begin{proof}
	\[ \dist(x^{n+p}, x^n) ≤ q \dist(x^{n+p-1}, x^{n-1}) ≤ \dotsb ≤ q^n \dist(x^p, x^0) \]
	По неравенству треугольника
	\[ \dist(x^p, x^0) ≤ \dist(x^p, x^{p-1}) + \dotsb + \dist(x^1, x^0) ≤ q^p + \dotsb q^0 ≤ \frac{1}{1 - q} \]
	Тогда
	\[ \dist(x^{n+p}, x^n) ≤ \frac{q^n}{1 - q} \]
	и по критерию Коши последовательность имеет предел $x$. Очевидно $g(x) = x$.
\end{proof}

\subsection{Метод Ньютона}
Пусть $F(x) = 0$ — система $m$ уравнений с $m$ неизвестными, а $\bar x$ — её корень. Тогда:
\[ 0 = F(\bar x) = F(x) + F'(x) (\bar x - x) + o(\bar x) \]
Соответствующий итерационный процесс:
\[ F'(x^n)(x^{n+1} - x^n) + F(x^n) = 0 \]
или
\[ x^{n+1} = x^n - F'(x^n)^{-1} F(x^n) \]

\begin{theorem}
	Пусть $F(\bar x) = 0$, и в $U_a(\bar x)$ выполнено:\begin{enumerate}
		\item $\norm{F'(x)^{-1}} ≤ a_1$
		\item $\norm{F(x) - F(y) - F'(y)(x - y)} ≤ a_2 \norm{x - y}^2$
	\end{enumerate}
	Тогда существует окрестность $U_b(\bar x)$, такая, что если $x^0 ∈ U_b(\bar x)$, то метод сходится.
\end{theorem}
\begin{proof}
	\begin{gather*}
		F'(x^n)(x^{n + 1} - \bar x) = F'(x^n) (x^{n + 1}) - x^n) + F'(x^n) (x^n - \bar x) = -F(x^n) + F'(x^n) (x^n - \bar x) \\
		x^{n+1} - \bar x = F'(x^n)^{-1} \left(F(\bar x) - F(x^n) + F'(x^n) (x^n - \bar x)\right) \\
		\norm{x^{n+1} - \bar x} ≤ \norm{F'(x^n)^{-1} (F(\bar x) - F(x^n) - F'(x^n) (\bar x - x^n)} ≤ a_1 a_2 \norm{\bar x - x^n}^2
	\end{gather*}
	Пусть $b < a$, $b < \frac{1}{a_1 a_2}$. Тогда:
	\[ a_1 a_2 \norm{x^{n+1} - \bar x} ≤ \left(a_1 a_2 \norm{x^{n} - \bar x} \right)^2 ≤ \dotsb ≤ \left(a_1 a_2 \norm{x^{n} - \bar x} \right)^{2^n} < q^{2^n} \]
	Таким образом, метод сходится, причём очень быстро. Такая сходимость называется суперсходимостью.
\end{proof}

\begin{example} Вычисление $\sqrt{2}$.

	Задача — решить уравнение $x^2 - 2 = 0$. Применим метод Ньютона:
	\[ x_{n+1} = x_n - \frac{x_n^2 - 2}{2 x_n} \,,\quad x_0 = 1.5 \]
	Имеем:
	\[ x_0 = 1.5,\, x_1 = 1.41\overbar{6}, x_2 = 1.4142156862\dotso \]
	То есть уже после 2 итераций получили 5 верных знаков после запятой.
\end{example}

\section{Численные методы решения задачи Коши для системы ОДУ}
\begin{example} Дифференциальное уравнение: $-u'' + u^3 = f(x)$, $u(0) = u(1) = 0$
	TODO
\end{example}

\end{document}
