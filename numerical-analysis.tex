\documentclass[a4paper,10pt,fleqn]{article}
\usepackage[margin=1in]{geometry}
\ifxetex
\usepackage{fontspec}
\usepackage{polyglossia}
\else
\usepackage[utf8]{inputenc}
\usepackage[T2A]{fontenc}
\usepackage[english,russian]{babel}
\fi
\usepackage{mathtools}
\usepackage{commath}
\usepackage{amscd,amsfonts,amssymb,amsthm}
\ifxetex
\usepackage{unicode-math}
\fi
%\usepackage{booktabs,multirow}
\usepackage{calc}
\usepackage{empheq}
\usepackage[inline]{enumitem}
\usepackage{framed}
\usepackage{graphicx}
\usepackage{titlesec}
\usepackage[space]{grffile}
\usepackage{hyperref}
\usepackage{lastpage}
\usepackage{setspace}
\usepackage[most]{tcolorbox}
\usepackage{xcolor}
%\usepackage[table]{xcolor}
%\usepackage{fancyhdr}
%\usepackage{mathrsfs}
%\usepackage{wrapfig}
%\usepackage{multicol}
%\usepackage{cancel}

\title{Лекции по численным методам}
\author{}
%\date{}
\ifxetex
\setmainfont{Nimbus Roman}
\setmathfont{texgyretermes-math.otf}
\setdefaultlanguage{russian}
\fi

\newcounter{lecture}
\newcommand{\lecture}[1]{{
	\stepcounter{lecture}
	\raggedleft{} \itshape
	Лекция \arabic{lecture} (#1)
	\hrule}}

\everymath{\displaystyle}
\hypersetup{
	colorlinks,
	citecolor=green,
	filecolor=red,
	linkcolor=red,
	urlcolor=blue
}

\renewcommand{\eval}{\Big\vert}
\newcommand{\const}{\operatorname{const}}
\DeclareMathOperator{\sgn}{sgn}
\DeclareMathOperator{\dist}{ρ}

\newcommand{\setw}[2]{\set{#1 \colon #2}}
\newcommand{\cvrg}[2]{\overset{#1}{\underset{#2}{\rightarrow}}}
\newcommand{\uncvrg}[2]{\overset{#1}{\underset{#2}{\rightrightarrows}}}
\newcommand{\uninscvrg}[2]{\uncvrg{\ins{#1}}{#2}}

% Inline list
\newlist{mylist}{enumerate*}{1}
\setlist[mylist]{label=(\arabic*)}

% Common environments
\theoremstyle{definition}
%\theoremstyle{plain}
\newtheorem{lemma}{Лемма}[section]
\newtheorem{theorem}{Теорема}[section]
\newtheorem{corollary}{Следствие}[section]
\theoremstyle{definition}
\newtheorem{definition}{Опредение}[section]
\newtheorem{proposition}{Предложение}[section]
\newtheorem{example}{Пример}[section]
\newtheorem{exercise}{Упражнение}[section]
\newtheorem*{exercise*}{Упражнение*}
\newtheorem{task}{Задача}[section]
\theoremstyle{remark}
\newtheorem*{note}{Замечание}

% Aliases
\newenvironment{df}{\begin{definition}}{\end{definition}}
\newenvironment{thm}{\begin{theorem}}{\end{theorem}}
\newenvironment{rk}{\begin{proposition}}{\end{proposition}}
\newenvironment{lm}{\begin{lemma}}{\end{lemma}}
\newenvironment{ex}{\begin{example}}{\end{example}}
\newenvironment{tk}{\begin{task}}{\end{task}}
\newenvironment{pf}{\begin{proof}}{\end{proof}}


\begin{document}
	\renewcommand{\setminus}{∖}
\abovedisplayskip=2mm
\belowdisplayskip=2mm
\setlength{\parindent}{0pt}
\setlength{\parskip}{3mm plus2mm minus1mm}
\newcommand{\eps}{\varepsilon}
\renewcommand{\phi}{\varphi}

	\maketitle
	\tableofcontents
	\pagebreak
	\lecture{12 февраля 2018 года}

\section{Решение нелинейных уравнений}

\subsection{Деление отрезка пополам}
Самый тривиальный метод, какой только можно придумать. Но сходится, только очень медленно.

\subsection{Метод простой итерации}
Пусть дано уравнение $x = g(x)$, причём $g$ — сжимающее отображение (т. е. $\dist(g(x), g(y)) \leq q\dist(x, y)$, где $q \in \intoo{0, 1}$). Рассмотрим последовательность $x^{n+1} = g(x^n)$ с начальным условием $x^0$. Она сходится к решению уравнения.
\begin{proof}
	\[ \dist(x^{n+p}, x^n) \leq q \dist(x^{n+p-1}, x^{n-1}) \leq \dotsb \leq q^n \dist(x^p, x^0) \]
	По неравенству треугольника
	\[ \dist(x^p, x^0) \leq \dist(x^p, x^{p-1}) + \dotsb + \dist(x^1, x^0) \leq q^p + \dotsb q^0 \leq \frac{1}{1 - q} \]
	Тогда
	\[ \dist(x^{n+p}, x^n) \leq \frac{q^n}{1 - q} \]
	и по критерию Коши последовательность имеет предел $x$. Очевидно $g(x) = x$.
\end{proof}

\subsection{Метод Ньютона}
Пусть $F(x) = 0$ — система $m$ уравнений с $m$ неизвестными, а $\bar x$ — её корень. Тогда:
\[ 0 = F(\bar x) = F(x) + F'(x) (\bar x - x) + o(\bar x) \]
Соответствующий итерационный процесс:
\[ F'(x^n)(x^{n+1} - x^n) + F(x^n) = 0 \]
или
\[ x^{n+1} = x^n - F'(x^n)^{-1} F(x^n) \]

\begin{theorem}
	Пусть $F(\bar x) = 0$, и в $U_a(\bar x)$ выполнено:\begin{enumerate}
		\item $\norm{F'(x)^{-1}} \leq a_1$
		\item $\norm{F(x) - F(y) - F'(y)(x - y)} \leq a_2 \norm{x - y}^2$
	\end{enumerate}
	Тогда существует окрестность $U_b(\bar x)$, такая, что если $x^0 \in U_b(\bar x)$, то метод сходится.
\end{theorem}
\begin{proof}
	\begin{gather*}
		F'(x^n)(x^{n + 1} - \bar x) = F'(x^n) (x^{n + 1}) - x^n) + F'(x^n) (x^n - \bar x) = -F(x^n) + F'(x^n) (x^n - \bar x) \\
		x^{n+1} - \bar x = F'(x^n)^{-1} \left(F(\bar x) - F(x^n) + F'(x^n) (x^n - \bar x)\right) \\
		\norm{x^{n+1} - \bar x} \leq \norm{F'(x^n)^{-1} (F(\bar x) - F(x^n) - F'(x^n) (\bar x - x^n)} \leq a_1 a_2 \norm{\bar x - x^n}^2
	\end{gather*}
	Пусть $b < a$, $b < \frac{1}{a_1 a_2}$. Тогда:
	\[ a_1 a_2 \norm{x^{n+1} - \bar x} \leq \left(a_1 a_2 \norm{x^{n} - \bar x} \right)^2 \leq \dotsb \leq \left(a_1 a_2 \norm{x^{n} - \bar x} \right)^{2^n} < q^{2^n} \]
	Таким образом, метод сходится, причём очень быстро. Такая сходимость называется суперсходимостью.
\end{proof}

\begin{example} Вычисление $\sqrt{2}$.

	Задача — решить уравнение $x^2 - 2 = 0$. Применим метод Ньютона:
	\[ x_{n+1} = x_n - \frac{x_n^2 - 2}{2 x_n} \,,\quad x_0 = 1.5 \]
	Имеем:
	\[ x_0 = 1.5,\, x_1 = 1.41\overline{6}, x_2 = 1.4142156862\dotso \]
	То есть уже после 2 итераций получили 5 верных знаков после запятой.
\end{example}

\begin{example} Дифференциальное уравнение: $-u'' + u^3 = f(x)$, $u(0) = u(1) = 0$
% TODO
\end{example}

\section{Численные методы решения задачи Коши для системы ОДУ}
% TODO
 % 2018-02-12 / [numberZero]
	\lecture{19 февраля 2018 года}

Пусть дана задача 
$$y'=f(x,y), \, y(0)=y_0. \qquad (1)$$
Нужно придумать методы, её решающие.\\
Допустим, что $y(x)$ известно. Тогда
$$y(x+h) = y(x) + \int_0^h f(x+t, y(x+t))dt = y(x) + hf(x, y(x)) + O(h^2).$$
Отсюда получаем формулу (\emph{метод Эйлера}):
$$y_{n+1} = y_n + hf(x_n, y_n). \qquad (2)$$
Всего $\frac{1}{h}$ шагов, погрешность такого метода на каждом шаге равна $O(h^2)$, погрешность на отрезке равна $O(h)$.
Возможен иной способ:
$$y(x+h) = y(x) + \frac{h}{2}\left( f(x, y(x)) + f(x+h, y(x+h)) \right) + O(h^3).$$
Впрочем, есть проблема, что $y(x+h)$ присутствует и слева, и справа, но здесь есть $h$. Поэтому из $y(x+h)=y^\ast+O(h^2)$ получаем
$$y(x+h) = y(x) + \frac{h}{2}\left( f(x, y(x)) + f(x+h, y^\ast) \right) + O(h^3),$$
$$\begin{cases}
	y^\ast = y_n + hf(x_n, y_n), \\
	y_{n+1} = y_n + \frac{h}{2}\left( f(x, y(x)) + f(x+h, y^\ast) \right).
\end{cases} \qquad (3)$$
Погрешность такого метода на каждом шаге уже $O(h^3)$, но зато значение $f$ вычисляется дважды.\\
Если взять квадратурную формулу прямоугольников, то получим
$$\begin{cases}
	y^\ast = y_n + \frac{h}{2}f(x_n, y_n), \\
	y_{n+1} = y_n + hf\left( x_n+\frac{h}{2}, y^\ast \right).
\end{cases} \qquad (4)$$
$(3), \, (4)$ --- \emph{модифицированный метод Эйлера}. Мы можем и дальше уточнять порядок пока это не устроит наши нужды.

\subsection{Метод Рунге--Кутты}
Сначала вычисляются поправки:
$$k_1(h) = hf(x, y),$$
$$k_2(h) = hf(x+\alpha_2h, y+\beta_{21}k_1),$$
$$\dots$$
$$k_q(h) = hf(x+\alpha_qh, y+\beta_{n1}k_1+\dots+\beta_{n,n-1}k_{n-1}), \qquad (n==q?)$$ % Тут точно всё так?
где $\alpha_i, \, \beta_i$ --- неизвестные коэффициенты, которые найдём потом.\\
$y(x+h) \approx z(h)=\sum_{j=1}^q p_jk_j(j)$. Хотим найти $\{ \alpha_2, \dots, \alpha_q \}$, матрицу 
$\begin{pmatrix}
	0          &       &                \\
	\beta_{21} & 0     &               &\\
	\vdots     &       & \text{\huge0}  \\
	\beta_{n1} & \dots & \beta_{n,n-1} & 0
\end{pmatrix}$ 
и $\left( p_1, \dots, p_1 \right).$
$$\phi(h) = y(x+h) - z(h) = \phi(0) + \phi'(0)h + \dots + \frac{\phi^{(s)}(0)}{s!}h^s + \frac{\phi^{(s+1)}(0)}{(s+1)!}h^{s+1} + O(h^{s+2}).$$
Хочется, чтобы $\phi(h)$ имело как можно больший порядок по $h$. Допустим, что $\phi(0) + \phi'(0)h + \dots + \frac{\phi^{(s)}(0)}{s!}h^s = 0$, а $\frac{\phi^{(s+1)}(0)}{(s+1)!}h^{s+1} \neq 0$. Тогда $s$ называется \emph{порядком метода}.
\begin{ex} \label{ex2-1}
	$q=1: \ \Rightarrow \ \text{формула Эйлера.}$
\end{ex}
\begin{ex} \label{ex2-2}
	$q=2: $
	$$\phi(h) = y(x+h) - y(x) - p_1k_1(h) - p_2k_2(h) = y(x+h) - y(x) - p_1hf(x,y) - p_2hf(x+\alpha_2h, y+\beta_{21}hf(x,y)).$$
	Разложим всё в ряд Тейлора:
	$$hf(x,y) + \frac{h^2}{2}\left( f_x - f_yf \right) + O(h^3) - p_1hf(x,y) - p_2h\left(f(x,\overline{x})+\alpha_2hf_x(x,\overline{y})+O(h^2)\right), \text{ где } \overline{y} = y+\beta_{21}hf(x,y).$$
	Продолжая вычисления и замены, получаем
	$$\phi(h) = hf + \frac{h^2}{2}\left( f_x + f_yf \right) - p_1hf - p_2hf - p_2\beta_{21}h^2ff_y - p_2\alpha_2h^2f_x + O(h^3).$$
	Из равенств $f-p_1f-p_2f=0$ и $\frac{f_x}{2}+\frac{f_yf}{2}-p_2\beta_{21}f_yf-p_2\alpha_2f_x=0$ получаем систему:
	$$\begin{cases}
		p_1 + p_2 = 0, \\
		p_2\alpha_2 = \frac{1}{2}, \\
		p_2\beta_{21} = \frac{1}{2}.
	\end{cases}$$
	При $p_1=p_2=\frac{1}{2} \ \Rightarrow \ \alpha_2=1, \, \beta_{21}=1$ --- получаем формулу $(3)$.\\
	При $p_1=0, \, p_2=1 \ \Rightarrow \ \alpha_2=\frac{1}{2}, \, \beta_{21}=\frac{1}{2}$ --- получаем формулу $(4)$.\\
	В нашем примере $s=2$.
\end{ex}
Заметим, что существует связь между $q$ и $s$: $q=2 \Rightarrow s=2, \, q=3 \Rightarrow s=3, \, q=4,5 \Rightarrow s=4$.\\
Итак, у нас квадратурная формула: $y_{n+1}=y_n+\frac{h}{2}\left( f(x_n,y_n)+f(x_{n+1},y_{n+1}) \right)$  --- сжимающее отображение. Поэтому для решения этого уравнения достаточно применить метод простой итерации:
$$y_{n+1}^0 = y_n + hf(x_n,y_n),$$
$$y_{n+1}^1 = y_n + \frac{h}{2}\left( f(x_n,y_n)+f(x_{n+1}, y_n+hf(x_n,y_n)) \right) \text{ --- совпадает с } (3).$$
Метод Рунге--Кутты работает хорошо, его недостаток --- отсутствие обобщения на случай уравнений с частными производными. Для того, чтобы применять этот метод нужно знать $y_0, \, f, \, \epsilon$ --- точность, с которой мы хотим получить результат, и $h_0$ --- начальный шаг.\\
Допустим, нам известно значение в точке $x$, хотим вычислить в точке $x+h$:
$$y(x+h) = y(x) + \dots + Mh^{s+1} + O(h^{s+2}).$$
Обозначим сумму $y(x) + \dots$ за $\tilde{y}(x+h)$ и вычислим этим же методом, но два раза с шагом $\frac{h}{2}$:
$$y(x+h) = \overline{y}(x+h) + 2M\left( \frac{h}{2} \right)^{s+1} + O(h^{s+2}).$$
Предположим, что мы находимся в зоне условия асимптотики, т.е. $O(h^{s+2})$ можно опустить:
$$\overline{y}(x+h) - \tilde{y}(x+h) = Mh^{s+1}\left( 1 - \frac{1}{2^s} \right).$$
Отсюда сможем выразить $Mh^{s+1}$, а это наша погрешность.
\begin{lm}[Критерий Рунге] \label{Runge-test}~\\
	Проверяем, выполняется ли $\left| \frac{\overline{y}-\tilde{y}}{1-\frac{1}{2^s}} \right| \leq \frac{\epsilon h}{l}$. Если нет, то делим $h$ пополам и повторяем.\\
	Если выполняется, то проверяем $\kappa\frac{\epsilon h}{l} \leq \left| \frac{\overline{y}-\tilde{y}}{1-\frac{1}{2^s}} \right|$. Если нет, то удваиваем $h$ и повторяем.
\end{lm}
Это называется \emph{методом Рунге--Кутты с автоматическим выбором шага}. Метод будет плохо работать, если функция резко прыгает вверх между точками, в которых считаем.

\subsection{Конечно-разностная схема}
Пусть дан отрезок, поровну поделённый на $h$ частей.
$$y'(x) \approx \frac{1}{h}\sum_{i=0}^q a_{-i}y(x-ih), \text{ где } a_{-i} \text{ --- коэффициенты, которые нужно отыскать}.$$
$$f(x, y(x)) = \sum_{i=0}^q b_{-i} f(x-ih, y(x-ih)),$$
$$\frac{1}{h}\sum_{i=0}^q a_{-i}y_{n-1} = \sum_{i=0}^q b_{-i}f(x_{n-i}, y_{n-1}). \qquad (5)$$
Чтобы считать по конечно-разностной схеме, нужно знать $y_0, \dots, y_{q-1}$. Их можно найти, например, методом Рунге--Кутты.
\begin{df} \label{df2-1}
	Функция 
	$$r(x) = \frac{1}{h}\sum_{i=0}^q a_{-i}y(x-ih) - \sum_{i=0}^q b_{-i}f(x_{n-i}, y_{n-1}), \text{ где } y(x-ih) \text{ --- точное решение } (1)$$
	называется \emph{погрешностью аппроксимации}.
\end{df}
\begin{ex} \label{ex2-3}
	Вычисление на квадрате.\\
	\textbf{\textit{TODO.}}
\end{ex} % 2018-02-19 / [arvego]
	\lecture{26 февраля 2018 года}

Рассматриваем уравнение
\[ y' = f(x, y) \]
с начальным условием
\[ y(0) = y_0\]
Сопоставим ему т. н. конечную разностную схему
\[ \frac{1}{h} \sum_{i=0}^{k} a_{-i} y_{n-i} - \sum_{i=0}^{k} b_{-i} f(x_{n-i}, y_{n-i}) = 0 \]
с начальными условиями $y_0, y_1, \dotsc, y_{k-1}$, где
\begin{gather*}
\frac{1}{h} \sum_{i=0}^{k} a_{-i} y_{n-i} \sim y' \\
\sum_{i=0}^{k} b_{-i} f(x_{n-i}, y_{n-i}) \sim f(x, y) \\
\end{gather*}
где $x = nh$. Разложим $y$ в ряд Тейлора по $h$:
\[ \frac{1}{h} \sum_{i=0}^{k} a_{-i} y(x - i h)
 = \sum_{i=0}^{k} a_{-i} \left( \frac{1}{h} y - i y' + \frac{h}{2} i^2 y'' - \dotsb \right)
 = \frac{1}{h} \sum_{i=0}^{k} a_{-i} y - \sum_{i=0}^{k} a_{-i} i y' + \frac{h}{2} \sum_{i=0}^{k} a_{-i} i^2 y'' - \dotsb
\]
Условия аппроксимации:
\begin{gather*}
\sum_{i=0}^{k} a_{-i} = 0 \\
\sum_{i=0}^{k} a_{-i} i=-1
\end{gather*}
Аналогично для правой части:
\[ \sum_{i=0}^{k} b_{-i} y'(x - i h)
 = \sum_{i=0}^{k} b_{-i} \left( y' - i h y'' + \dotsb \right)
 = \sum_{i=0}^{k} b_{-i} f(x, y) - \sum_{i=0}^{k} b_{-i} i h y'' + \dotsb
\]
Условие аппроксимации:
\[ \sum_{i=0}^{k} b_{-i} = 1 \]
Погрешность аппроксимации:
\[ r(x)
 = \frac{1}{h} \sum_{i=0}^{k} a_{-i} y(x - ih) - \sum_{i=0}^{k} b_{-i} f(x - ih, y(x - ih))
 = \frac{1}{h} \sum_{i=0}^{k} a_{-i} y(x - ih) - \sum_{i=0}^{k} b_{-i} y'(x - ih)
\]
где $y$ — точное решение.

Решаем разностную схему в различных случаях:
\begin{enumerate}
	\item $a_0 ≠ 0, b_0 = 0$ — явная разностная схема: считаем напрямую;
	\item $a_0 ≠ 0, b_0 ≠ 0$ — неявная разностная схема: решаем систему уравнений;
	\item $a_0 = 0, b_0 ≠ 0$ — схема с забеганием вперёд; такие не используем.
\end{enumerate}

\subsection{Подбор коэффициентов}
\begin{definition}
	Разностная схема имеет $m$-й порядок аппроксимации, если $r(x) = O(h^m)$ при $h → 0$.
\end{definition}
Хотим получить максимальное $m$. Считая количество уравнений и неизвестных, получаем, что можно получить порядок аппроксимации около $2k$. Однако, реальность несколько хуже.
\begin{example}
	Рассмотрим уравнение
	\[ y' = 0 \]
	Его решения хорошо известны. Это $y = y_0 = \const$. Разностная схема же принимает вид:
	\[ \sum_{i=0}^{k} a_{-i} y_{n-i} = 0 \]
	Пусть значения $y_i$ заданы с погрешностью $ε_i$. Тогда уравнение на погрешность:
	\[ \sum_{i=0}^{k} a_{-i} ε_{n-i} = 0 \]
	Это уравнение в конечных разностях. Ищем решение в виде $ε_n = μ^n$.
	\[ \sum_{i=0}^{k} a_{-i} μ^{n-i} = 0 \]
	При $n = k$ получаем т. н. характеристическое уравнение разностной схемы:
	\[ \sum_{i=0}^{k} a_{-i} μ^{k-i} = 0 \]
	Рассмотрим плохие случаи:
	\begin{enumerate}
		\item Существует корень $μ_1 ∈ ℝ, \abs{μ_1} > 1$. Тогда $ε_n = C_1 μ_1^n + \dotsb$ растёт \emph{экспоненциально}.
		\item Существует корень $μ_1 ∈ ℂ, \abs{μ_1} > 1$. Тогда $ε_n = C_1 ρ^n \cos nφ + C_2 ρ^n \sin nφ + \dotsb$ — тоже.
		\item Существует кратный корень $μ_1$, кратности $q$, $\abs{μ_1} = 1$. Тогда:
		$ε_n = \left( C_0 + C_1 n + \dotsb + C_{q-1} n^{q-1} \right) μ_1^n$ — растёт полиномиально. Однако уже для $y' = y$ рост погрешности снова экспоненциальный.
	\end{enumerate}
	Остальные случаи хорошие. Отсюда получаем
\end{example}
\begin{claim}[Условие α] Для того, чтобы решение было устойчиво, необходимо, чтобы все корни $\sum_{i=0}^{k} a_{-i} μ^{k-i} = 0$ лежали в круге $\abs{z} ≤ 1$, и на границе не было кратных корней.
\end{claim}
С таким условием можно получить порядок аппроксимации $m$ лишь около $k$:
\begin{itemize}
	\item Если схема явная и $m > k$, то условие α не выполняется.
	\item Если схема явная, то, если $m$ нечётно и $m > k + 1$, или если $m > k + 2$, то α не выполняется.
\end{itemize}

\begin{example}
	Рассмотрим уравнение
	\[ y' + My = 0 \]
	Решение:
	\[ y(x) = y_0 \exp{-Mx} \]
	При $M > 0$ точное решение монотонно убывает.
	
	Рассмотрим метод Эйлера:
	\[ y_{n+1} = y_n - h M y_n \]

%TODO
\end{example}
 % 2018-02-26 / [numberZero]
\end{document}
