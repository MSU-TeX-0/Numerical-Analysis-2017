\lecture{12 февраля 2018 года}

\section{Решение нелинейных уравнений}

\subsection{Деление отрезка пополам}
Самый тривиальный метод, какой только можно придумать. Но сходится, только очень медленно.

\subsection{Метод простой итерации}
Пусть дано уравнение $x = g(x)$, причём $g$ — сжимающее отображение (т. е. $\dist(g(x), g(y)) \leq q\dist(x, y)$, где $q \in \intoo{0, 1}$). Рассмотрим последовательность $x^{n+1} = g(x^n)$ с начальным условием $x^0$. Она сходится к решению уравнения.
\begin{proof}
	\[ \dist(x^{n+p}, x^n) \leq q \dist(x^{n+p-1}, x^{n-1}) \leq \dotsb \leq q^n \dist(x^p, x^0) \]
	По неравенству треугольника
	\[ \dist(x^p, x^0) \leq \dist(x^p, x^{p-1}) + \dotsb + \dist(x^1, x^0) \leq q^p + \dotsb q^0 \leq \frac{1}{1 - q} \]
	Тогда
	\[ \dist(x^{n+p}, x^n) \leq \frac{q^n}{1 - q} \]
	и по критерию Коши последовательность имеет предел $x$. Очевидно $g(x) = x$.
\end{proof}

\subsection{Метод Ньютона}
Пусть $F(x) = 0$ — система $m$ уравнений с $m$ неизвестными, а $\bar x$ — её корень. Тогда:
\[ 0 = F(\bar x) = F(x) + F'(x) (\bar x - x) + o(\bar x) \]
Соответствующий итерационный процесс:
\[ F'(x^n)(x^{n+1} - x^n) + F(x^n) = 0 \]
или
\[ x^{n+1} = x^n - F'(x^n)^{-1} F(x^n) \]

\begin{theorem}
	Пусть $F(\bar x) = 0$, и в $U_a(\bar x)$ выполнено:\begin{enumerate}
		\item $\norm{F'(x)^{-1}} \leq a_1$
		\item $\norm{F(x) - F(y) - F'(y)(x - y)} \leq a_2 \norm{x - y}^2$
	\end{enumerate}
	Тогда существует окрестность $U_b(\bar x)$, такая, что если $x^0 \in U_b(\bar x)$, то метод сходится.
\end{theorem}
\begin{proof}
	\begin{gather*}
		F'(x^n)(x^{n + 1} - \bar x) = F'(x^n) (x^{n + 1}) - x^n) + F'(x^n) (x^n - \bar x) = -F(x^n) + F'(x^n) (x^n - \bar x) \\
		x^{n+1} - \bar x = F'(x^n)^{-1} \left(F(\bar x) - F(x^n) + F'(x^n) (x^n - \bar x)\right) \\
		\norm{x^{n+1} - \bar x} \leq \norm{F'(x^n)^{-1} (F(\bar x) - F(x^n) - F'(x^n) (\bar x - x^n)} \leq a_1 a_2 \norm{\bar x - x^n}^2
	\end{gather*}
	Пусть $b < a$, $b < \frac{1}{a_1 a_2}$. Тогда:
	\[ a_1 a_2 \norm{x^{n+1} - \bar x} \leq \left(a_1 a_2 \norm{x^{n} - \bar x} \right)^2 \leq \dotsb \leq \left(a_1 a_2 \norm{x^{n} - \bar x} \right)^{2^n} < q^{2^n} \]
	Таким образом, метод сходится, причём очень быстро. Такая сходимость называется суперсходимостью.
\end{proof}

\begin{example} Вычисление $\sqrt{2}$.

	Задача — решить уравнение $x^2 - 2 = 0$. Применим метод Ньютона:
	\[ x_{n+1} = x_n - \frac{x_n^2 - 2}{2 x_n} \,,\quad x_0 = 1.5 \]
	Имеем:
	\[ x_0 = 1.5,\, x_1 = 1.41\overline{6}, x_2 = 1.4142156862\dotso \]
	То есть уже после 2 итераций получили 5 верных знаков после запятой.
\end{example}

\begin{example} Дифференциальное уравнение: $-u'' + u^3 = f(x)$, $u(0) = u(1) = 0$
% TODO
\end{example}

\section{Численные методы решения задачи Коши для системы ОДУ}
% TODO
