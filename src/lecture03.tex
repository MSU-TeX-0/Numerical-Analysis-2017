\lecture{26 февраля 2018 года}

Рассматриваем уравнение
\[ y' = f(x, y) \]
с начальным условием
\[ y(0) = y_0\]
Сопоставим ему т. н. конечную разностную схему
\[ \frac{1}{h} \sum_{i=0}^{k} a_{-i} y_{n-i} - \sum_{i=0}^{k} b_{-i} f(x_{n-i}, y_{n-i}) = 0 \]
с начальными условиями $y_0, y_1, \dotsc, y_{k-1}$, где
\begin{gather*}
\frac{1}{h} \sum_{i=0}^{k} a_{-i} y_{n-i} \sim y' \\
\sum_{i=0}^{k} b_{-i} f(x_{n-i}, y_{n-i}) \sim f(x, y) \\
\end{gather*}
где $x = nh$. Разложим $y$ в ряд Тейлора по $h$:
\[ \frac{1}{h} \sum_{i=0}^{k} a_{-i} y(x - i h)
 = \sum_{i=0}^{k} a_{-i} \left( \frac{1}{h} y - i y' + \frac{h}{2} i^2 y'' - \dotsb \right)
 = \frac{1}{h} \sum_{i=0}^{k} a_{-i} y - \sum_{i=0}^{k} a_{-i} i y' + \frac{h}{2} \sum_{i=0}^{k} a_{-i} i^2 y'' - \dotsb
\]
Условия аппроксимации:
\begin{gather*}
\sum_{i=0}^{k} a_{-i} = 0 \\
\sum_{i=0}^{k} a_{-i} i=-1
\end{gather*}
Аналогично для правой части:
\[ \sum_{i=0}^{k} b_{-i} y'(x - i h)
 = \sum_{i=0}^{k} b_{-i} \left( y' - i h y'' + \dotsb \right)
 = \sum_{i=0}^{k} b_{-i} f(x, y) - \sum_{i=0}^{k} b_{-i} i h y'' + \dotsb
\]
Условие аппроксимации:
\[ \sum_{i=0}^{k} b_{-i} = 1 \]
Погрешность аппроксимации:
\[ r(x)
 = \frac{1}{h} \sum_{i=0}^{k} a_{-i} y(x - ih) - \sum_{i=0}^{k} b_{-i} f(x - ih, y(x - ih))
 = \frac{1}{h} \sum_{i=0}^{k} a_{-i} y(x - ih) - \sum_{i=0}^{k} b_{-i} y'(x - ih)
\]
где $y$ — точное решение.

Решаем разностную схему в различных случаях:
\begin{enumerate}
	\item $a_0 ≠ 0, b_0 = 0$ — явная разностная схема: считаем напрямую;
	\item $a_0 ≠ 0, b_0 ≠ 0$ — неявная разностная схема: решаем систему уравнений;
	\item $a_0 = 0, b_0 ≠ 0$ — схема с забеганием вперёд; такие не используем.
\end{enumerate}

\subsection{Подбор коэффициентов}
\begin{definition}
	Разностная схема имеет $m$-й порядок аппроксимации, если $r(x) = O(h^m)$ при $h → 0$.
\end{definition}
Хотим получить максимальное $m$. Считая количество уравнений и неизвестных, получаем, что можно получить порядок аппроксимации около $2k$. Однако, реальность несколько хуже.
\begin{example}
	Рассмотрим уравнение
	\[ y' = 0 \]
	Его решения хорошо известны. Это $y = y_0 = \const$. Разностная схема же принимает вид:
	\[ \sum_{i=0}^{k} a_{-i} y_{n-i} = 0 \]
	Пусть значения $y_i$ заданы с погрешностью $ε_i$. Тогда уравнение на погрешность:
	\[ \sum_{i=0}^{k} a_{-i} ε_{n-i} = 0 \]
	Это уравнение в конечных разностях. Ищем решение в виде $ε_n = μ^n$.
	\[ \sum_{i=0}^{k} a_{-i} μ^{n-i} = 0 \]
	При $n = k$ получаем т. н. характеристическое уравнение разностной схемы:
	\[ \sum_{i=0}^{k} a_{-i} μ^{k-i} = 0 \]
	Рассмотрим плохие случаи:
	\begin{enumerate}
		\item Существует корень $μ_1 ∈ ℝ, \abs{μ_1} > 1$. Тогда $ε_n = C_1 μ_1^n + \dotsb$ растёт \emph{экспоненциально}.
		\item Существует корень $μ_1 ∈ ℂ, \abs{μ_1} > 1$. Тогда $ε_n = C_1 ρ^n \cos nφ + C_2 ρ^n \sin nφ + \dotsb$ — тоже.
		\item Существует кратный корень $μ_1$, кратности $q$, $\abs{μ_1} = 1$. Тогда:
		$ε_n = \left( C_0 + C_1 n + \dotsb + C_{q-1} n^{q-1} \right) μ_1^n$ — растёт полиномиально. Однако уже для $y' = y$ рост погрешности снова экспоненциальный.
	\end{enumerate}
	Остальные случаи хорошие. Отсюда получаем
\end{example}
\begin{claim}[Условие α] Для того, чтобы решение было устойчиво, необходимо, чтобы все корни $\sum_{i=0}^{k} a_{-i} μ^{k-i} = 0$ лежали в круге $\abs{z} ≤ 1$, и на границе не было кратных корней.
\end{claim}
С таким условием можно получить порядок аппроксимации $m$ лишь около $k$:
\begin{itemize}
	\item Если схема явная и $m > k$, то условие α не выполняется.
	\item Если схема явная, то, если $m$ нечётно и $m > k + 1$, или если $m > k + 2$, то α не выполняется.
\end{itemize}

\begin{example}
	Рассмотрим уравнение
	\[ y' + My = 0 \]
	Решение:
	\[ y(x) = y_0 \exp{-Mx} \]
	При $M > 0$ точное решение монотонно убывает.
	
	Рассмотрим метод Эйлера:
	\[ y_{n+1} = y_n - h M y_n \]

%TODO
\end{example}
