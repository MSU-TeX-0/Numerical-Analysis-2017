\lecture{19 февраля 2018 года}

Пусть дана задача 
$$y'=f(x,y), \, y(0)=y_0. \qquad (1)$$
Нужно придумать методы, её решающие.\\
Допустим, что $y(x)$ известно. Тогда
$$y(x+h) = y(x) + \int_0^h f(x+t, y(x+t))dt = y(x) + hf(x, y(x)) + O(h^2).$$
Отсюда получаем формулу (\emph{метод Эйлера}):
$$y_{n+1} = y_n + hf(x_n, y_n). \qquad (2)$$
Всего $\frac{1}{h}$ шагов, погрешность такого метода на каждом шаге равна $O(h^2)$, погрешность на отрезке равна $O(h)$.
Возможен иной способ:
$$y(x+h) = y(x) + \frac{h}{2}\left( f(x, y(x)) + f(x+h, y(x+h)) \right) + O(h^3).$$
Впрочем, есть проблема, что $y(x+h)$ присутствует и слева, и справа, но здесь есть $h$. Поэтому из $y(x+h)=y^\ast+O(h^2)$ получаем
$$y(x+h) = y(x) + \frac{h}{2}\left( f(x, y(x)) + f(x+h, y^\ast) \right) + O(h^3),$$
$$\begin{cases}
	y^\ast = y_n + hf(x_n, y_n), \\
	y_{n+1} = y_n + \frac{h}{2}\left( f(x, y(x)) + f(x+h, y^\ast) \right).
\end{cases} \qquad (3)$$
Погрешность такого метода на каждом шаге уже $O(h^3)$, но зато значение $f$ вычисляется дважды.\\
Если взять квадратурную формулу прямоугольников, то получим
$$\begin{cases}
	y^\ast = y_n + \frac{h}{2}f(x_n, y_n), \\
	y_{n+1} = y_n + hf\left( x_n+\frac{h}{2}, y^\ast \right).
\end{cases} \qquad (4)$$
$(3), \, (4)$ --- \emph{модифицированный метод Эйлера}. Мы можем и дальше уточнять порядок пока это не устроит наши нужды.

\subsection{Метод Рунге--Кутты}
Сначала вычисляются поправки:
$$k_1(h) = hf(x, y),$$
$$k_2(h) = hf(x+\alpha_2h, y+\beta_{21}k_1),$$
$$\dots$$
$$k_q(h) = hf(x+\alpha_qh, y+\beta_{n1}k_1+\dots+\beta_{n,n-1}k_{n-1}), \qquad (n==q?)$$ % Тут точно всё так?
где $\alpha_i, \, \beta_i$ --- неизвестные коэффициенты, которые найдём потом.\\
$y(x+h) \approx z(h)=\sum_{j=1}^q p_jk_j(j)$. Хотим найти $\{ \alpha_2, \dots, \alpha_q \}$, матрицу 
$\begin{pmatrix}
	0          &       &                \\
	\beta_{21} & 0     &               &\\
	\vdots     &       & \text{\huge0}  \\
	\beta_{n1} & \dots & \beta_{n,n-1} & 0
\end{pmatrix}$ 
и $\left( p_1, \dots, p_1 \right).$
$$\phi(h) = y(x+h) - z(h) = \phi(0) + \phi'(0)h + \dots + \frac{\phi^{(s)}(0)}{s!}h^s + \frac{\phi^{(s+1)}(0)}{(s+1)!}h^{s+1} + O(h^{s+2}).$$
Хочется, чтобы $\phi(h)$ имело как можно больший порядок по $h$. Допустим, что $\phi(0) + \phi'(0)h + \dots + \frac{\phi^{(s)}(0)}{s!}h^s = 0$, а $\frac{\phi^{(s+1)}(0)}{(s+1)!}h^{s+1} \neq 0$. Тогда $s$ называется \emph{порядком метода}.
\begin{ex} \label{ex2-1}
	$q=1: \ \Rightarrow \ \text{формула Эйлера.}$
\end{ex}
\begin{ex} \label{ex2-2}
	$q=2: $
	$$\phi(h) = y(x+h) - y(x) - p_1k_1(h) - p_2k_2(h) = y(x+h) - y(x) - p_1hf(x,y) - p_2hf(x+\alpha_2h, y+\beta_{21}hf(x,y)).$$
	Разложим всё в ряд Тейлора:
	$$hf(x,y) + \frac{h^2}{2}\left( f_x - f_yf \right) + O(h^3) - p_1hf(x,y) - p_2h\left(f(x,\overline{x})+\alpha_2hf_x(x,\overline{y})+O(h^2)\right), \text{ где } \overline{y} = y+\beta_{21}hf(x,y).$$
	Продолжая вычисления и замены, получаем
	$$\phi(h) = hf + \frac{h^2}{2}\left( f_x + f_yf \right) - p_1hf - p_2hf - p_2\beta_{21}h^2ff_y - p_2\alpha_2h^2f_x + O(h^3).$$
	Из равенств $f-p_1f-p_2f=0$ и $\frac{f_x}{2}+\frac{f_yf}{2}-p_2\beta_{21}f_yf-p_2\alpha_2f_x=0$ получаем систему:
	$$\begin{cases}
		p_1 + p_2 = 0, \\
		p_2\alpha_2 = \frac{1}{2}, \\
		p_2\beta_{21} = \frac{1}{2}.
	\end{cases}$$
	При $p_1=p_2=\frac{1}{2} \ \Rightarrow \ \alpha_2=1, \, \beta_{21}=1$ --- получаем формулу $(3)$.\\
	При $p_1=0, \, p_2=1 \ \Rightarrow \ \alpha_2=\frac{1}{2}, \, \beta_{21}=\frac{1}{2}$ --- получаем формулу $(4)$.\\
	В нашем примере $s=2$.
\end{ex}
Заметим, что существует связь между $q$ и $s$: $q=2 \Rightarrow s=2, \, q=3 \Rightarrow s=3, \, q=4,5 \Rightarrow s=4$.\\
Итак, у нас квадратурная формула: $y_{n+1}=y_n+\frac{h}{2}\left( f(x_n,y_n)+f(x_{n+1},y_{n+1}) \right)$  --- сжимающее отображение. Поэтому для решения этого уравнения достаточно применить метод простой итерации:
$$y_{n+1}^0 = y_n + hf(x_n,y_n),$$
$$y_{n+1}^1 = y_n + \frac{h}{2}\left( f(x_n,y_n)+f(x_{n+1}, y_n+hf(x_n,y_n)) \right) \text{ --- совпадает с } (3).$$
Метод Рунге--Кутты работает хорошо, его недостаток --- отсутствие обобщения на случай уравнений с частными производными. Для того, чтобы применять этот метод нужно знать $y_0, \, f, \, \epsilon$ --- точность, с которой мы хотим получить результат, и $h_0$ --- начальный шаг.\\
Допустим, нам известно значение в точке $x$, хотим вычислить в точке $x+h$:
$$y(x+h) = y(x) + \dots + Mh^{s+1} + O(h^{s+2}).$$
Обозначим сумму $y(x) + \dots$ за $\tilde{y}(x+h)$ и вычислим этим же методом, но два раза с шагом $\frac{h}{2}$:
$$y(x+h) = \overline{y}(x+h) + 2M\left( \frac{h}{2} \right)^{s+1} + O(h^{s+2}).$$
Предположим, что мы находимся в зоне условия асимптотики, т.е. $O(h^{s+2})$ можно опустить:
$$\overline{y}(x+h) - \tilde{y}(x+h) = Mh^{s+1}\left( 1 - \frac{1}{2^s} \right).$$
Отсюда сможем выразить $Mh^{s+1}$, а это наша погрешность.
\begin{lm}[Критерий Рунге] \label{Runge-test}~\\
	Проверяем, выполняется ли $\left| \frac{\overline{y}-\tilde{y}}{1-\frac{1}{2^s}} \right| \leq \frac{\epsilon h}{l}$. Если нет, то делим $h$ пополам и повторяем.\\
	Если выполняется, то проверяем $\kappa\frac{\epsilon h}{l} \leq \left| \frac{\overline{y}-\tilde{y}}{1-\frac{1}{2^s}} \right|$. Если нет, то удваиваем $h$ и повторяем.
\end{lm}
Это называется \emph{методом Рунге--Кутты с автоматическим выбором шага}. Метод будет плохо работать, если функция резко прыгает вверх между точками, в которых считаем.

\subsection{Конечно-разностная схема}
Пусть дан отрезок, поровну поделённый на $h$ частей.
$$y'(x) \approx \frac{1}{h}\sum_{i=0}^q a_{-i}y(x-ih), \text{ где } a_{-i} \text{ --- коэффициенты, которые нужно отыскать}.$$
$$f(x, y(x)) = \sum_{i=0}^q b_{-i} f(x-ih, y(x-ih)),$$
$$\frac{1}{h}\sum_{i=0}^q a_{-i}y_{n-1} = \sum_{i=0}^q b_{-i}f(x_{n-i}, y_{n-1}). \qquad (5)$$
Чтобы считать по конечно-разностной схеме, нужно знать $y_0, \dots, y_{q-1}$. Их можно найти, например, методом Рунге--Кутты.
\begin{df} \label{df2-1}
	Функция 
	$$r(x) = \frac{1}{h}\sum_{i=0}^q a_{-i}y(x-ih) - \sum_{i=0}^q b_{-i}f(x_{n-i}, y_{n-1}), \text{ где } y(x-ih) \text{ --- точное решение } (1)$$
	называется \emph{погрешностью аппроксимации}.
\end{df}
\begin{ex} \label{ex2-3}
	Вычисление на квадрате.\\
	\textbf{\textit{TODO.}}
\end{ex}